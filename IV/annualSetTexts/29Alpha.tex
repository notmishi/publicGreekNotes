% vimTeX root = .
\documentclass{article}
\usepackage{fontspec}
\usepackage{polyglossia}

\setmainlanguage{english}
\setotherlanguage[variant=polytonic]{greek}
\newfontfamily\greekfont[Script=Greek]{Gentium Plus}

\begin{document}

\section{Preface}

Greek text is sourced from \emph{Athenaze Book II, Second Edition} (2003) \\
English translation is adapted from \emph{Athenaze Book II: Teacher's Handbook, Revised Edition} (1991) \\

This chapter contains an English passage within the Greek text, after the first paragraph:\\
\em In the summer of 429 BC, 
a Corinthian fleet of forty-seven ships tried to slip through Phormio's blockade
and take reinforcements to their allies when fighting in Acarnania in northwest Greece.
\em

\section{Chapter 29\textgreek{α}}

\subsection*{\textgreek{ΜΕΓΑ ΤΟ ΤΗΣ ΘΑΛΑΣΣΗΣ ΚΡΑΤΟΣ}}
\subsection*{Great is the Power of the Sea}

\begin{greek}
τοῦ δὲ ἐπιγιγνομένου χειμῶνος Ἀθηναῖοι ναῦς ἔστειλαν εἴκοσι μὲν περὶ Πελοπόννησον
καὶ Φορμίωνα στρατηγόν, ὃς ὁρμώμενος ἐκ Ναυπάκτου φυλακὴν εἶχεν ὥστε μήτ᾿ ἐκπλεῖν ἐκ Κορίνθου
καὶ τοῦ Κρισαίου κόλπου μηδένα μήτ᾿ ἐσπλεῖν. \\
\end{greek}


The following winter, the Athenians sent twenty ships around the Peloponnesus
with Phormio as the general, who, based on Naupactus, kept guard
so that no one should sail in or out of Corinth and the Gulf of Corinth. \\ % kinda wordy


\begin{greek}
οἱ δὲ Κορίνθιοι καὶ οἱ ἄλλοι ξύμμαχοι ἠναγκάσθησαν
περὶ τὰς αὐτὰς ἡμέρας ναυμαχῆσαι πρὸς Φορμίωνα
καὶ τὰς εἴκοσι ναῦς τῶν Ἀθηναίων αἳ ἐφρούρουν ἐν Ναυπάκτῳ.
ὁ γὰρ Φορμίων παραπλέοντας αὐτοὺς ἔξω τοῦ κόλπου ἐτήρει,
βουλόμενος ἐν τῇ εὐρυχωρίᾳ ἐπιθέσθαι. \\
\end{greek}


The Corinthians and their allies were compelled to fight a sea battle about this time
against Phormio and the twenty Athenian ships that were on guard at Naupactus.
For Phormio was watching them as they were sailing along outside the gulf,
wanting to attack them in open waters. \\


\begin{greek}
οἱ δὲ Κορίνθιοι καὶ οἱ σύμμαχοι ἔπλεον μὲν οὐχ ὡς ἐπὶ ναυμαχίᾳ
ἀλλὰ στρατιωτικώτερον παρεσκευασμένοι ἐς τὴν Ἀκαρνανίαν,
καὶ οὐκ οἰόμενοι τοὺς Ἀθηναίους ἂν τολμῆσαι ναυμαχίαν ποιήσασθαι·
παρὰ γῆν σφῶν μέντοι κομιζόμενοι τοὺς Ἀθηναίους ἀντιπαραπλέοντας ἑώρων καί,
ἐπεὶ ἐκ Πατρῶν τῆς Ἀχαΐας πρὸς τὴν ἀντιπέρας ἤπειρον διέβαλλον,
εἶδον τοὺς Ἀθηναίους ἀπὸ Χαλκίδος προσπλέοντας σφίσιν·
οὕτω δὴ ἀναγκάζονται ναυμαχεῖν κατὰ μέσον τὸν πορθμόν. \\
\end{greek}


The Corinthians and their allies were sailing,
prepared not for battle but more for carrying troops to Acarnania,
and they did not think that the Athenians would dare start a naval battle.
But sailing past their own land, they saw the Athenians sailing along opposite,
and when they were crossing from Patrae in Achaea toward the mainland opposite,
they saw the Athenians sailing toward them from Chalcis.
So they are compelled to fight in the middle of the straits. \\ % why is there change of tense ??


\begin{greek}
καὶ οἱ μὲν Πελοππονήσιοι ἐτάξαντο κύκλον τῶν νεῶν ὡς μέγιστον οἷοί τ᾿ ἦσαν,
τὰς πρῴρας μὲν ἔξω, ἔσω δὲ τὰς πρύμνας, καὶ τὰ λεπτὰ πλοῖα ἃ ξυνέπλει ἐντὸς ποιοῦνται.
οἱ δὲ Ἀθηναῖοι κατὰ μίαν ναῦν τεταγμένοι περιέπλεον αὐτοὺς κύκλῳ καὶ ξυνῆγον ἐς ὀλίγον,
ἐν χρῷ αἰεὶ παραπλέοντες·
προείρητο δ᾿ αὐτοῖς ὑπὸ Φορμίωνος μὴ ἐπιχειρεῖν πρὶν ἂν αὐτὸς σημήνῃ.
ἤλπιζε γὰρ αὐτῶν οὐ μενεῖν τὴν τάξιν ἀλλὰ τὰς ναῦς ξυμπεσεῖσθαι πρὸς ἀλλήλας
καὶ τὰ πλοῖα ταραχὴν παρέξειν· εἴ τ᾿ ἐκπνεύσειεν ἐκ τοῦ κόλπου τὸ πνεῦμα,
ὅπερ εἰώθει γίγνεσθαι ἐπὶ τὴν ἕω, οὐδένα χρόνον ἡσυχάσειν αὐτούς. \\
\end{greek}



And the Peloponnesians formed a circle of their ships, as large as they were able,
the prows facing outward and the sterns inward, and they put the light boats,
which were sailing with them, inside.
And the Athenians, drawn up in single file, sailed around them in a circle
and compressed them into a small space, always sailing by within a hair's breadth;
an order had been given to them by Phormio beforehand that they were not to attack until he gave the signal.
For he expected that their formation would not hold
but that their ships would crash into each other and that the boats would cause confusion;
and if the breeze blew out of the gulf, which usually happened toward dawn,
they would not keep their formation for any time. \\ % the 'for any time' irks me but idk what i would replace it for

\end{document}
