% vimTeX root = .
\documentclass{article}
\usepackage{fontspec}
\usepackage{polyglossia}
\usepackage[left=1.5in, right = 1.5in, top = 1.5in, bottom=1.25in]{geometry} % last line was getting pushed to next page

\setmainlanguage{english}
\setotherlanguage[variant=polytonic]{greek}
\newfontfamily\greekfont[Script=Greek]{Gentium Plus}

\begin{document}

\section{Preface}

Greek text is sourced from \emph{Athenaze Book II, Second Edition} (2003) \\
English translation is adapted from \emph{Athenaze Book II: Teacher's Handbook, Revised Edition} (1991) \\

\section{Chapter 29\textgreek{γ}}

\subsection*{\textgreek{ΜΕΓΑ ΤΟ ΤΗΣ ΘΑΛΑΣΣΗΣ ΚΡΑΤΟΣ}}
\subsection*{Great is the Power of the Sea}

\begin{greek}
οἱ δὲ ἐν τῇ Κυλλήνῃ Πελοποννήσιοι, ἐν ᾧ οἱ Ἀθηναῖοι περὶ τὴν Κρήτην κατείχοντο,
παρεσκευασμένοι ὡς ἐπὶ ναυμαχίαν παρέπλευσαν ἐς Πάνορμον τὸν Ἀχαϊκόν,
οὗπερ αὐτοῖς ὁ κατὰ γῆν στρατὸς τῶν Πελοποννησίων προσεβεβοηθήκει.
παρέπλευσε δὲ καὶ ὁ Φορμίων ἐπὶ τὸ Ῥίον τὸ Μολυκρικόν, καὶ ὡρμίσατο ἔξω αὐτοῦ ναυσὶν εἴκοσι, αἷσπερ καὶ ἐναυμάχησεν.
ἐπὶ οὖν τῷ Ῥίῳ τῷ Ἀχαϊκῷ οἱ Πελοποννήσιοι, ἀπέχοντι οὐ πολὺ τοῦ Πανόρμου,
ὡρμίσαντο καὶ αὐτοὶ ναυσὶν ἑπτὰ καὶ ἑβδομήκοντα, ἐπειδὴ καὶ τοὺς Ἀθηναίους εἶδον. \\
\end{greek}


The Peloponnesians in Cyllene, while the Athenians were held back around Crete,
sailed ready for battle to Panormus of Achaea,
where the land force of the Peloponnesians had come to their aid.
And Phormio also sailed along to Molycrian Rhion and anchored outside it with the twenty ships,
with which he had already fought.
The Peloponnesians themselves also came to anchor at Rhion in Achaea,
not far from Panormus, with seventy-seven ships, when they actually saw the Athenians. \\


\begin{greek}
καὶ ἐπὶ μὲν ἓξ ἢ ἑπτὰ ἡμέρας ἀνθώρμουν ἀλλήλοις,
μελετῶντές τε καὶ παρασκευαζόμενοι τὴν ναυμαχίαν,
γνώμην ἔχοντες οἱ μὲν Πελοποννήσιοι μὴ ἐκπλεῖν ἔξω τῶν Ῥίων ἐς τὴν εὐρυχωρίαν,
φοβούμενοι τὸ πρότερον πάθος, οἱ δὲ Ἀθηναῖοι μὴ ἐσπλεῖν ἐς τὰ στενά,
νομίζοντες πρὸς ἐκείνων εἶναι τὴν ἐν ὀλίγῳ ναυμαχίαν.
ἔπειτα ὁ Κνῆμος καὶ οἱ ἄλλοι τῶν Πελοποννησίων στρατηγοί, βουλόμενοι ταχέως τὴν ναυμαχίαν ποιῆσαι,
πρίν τι καὶ ἀπὸ τῶν Ἀθηναίων ἐπιβοηθῆσαι, ξυνεκάλεσαν τοὺς στρατιώτας,
καὶ ὁρῶντες αὐτῶν τοὺς πολλοὺς διὰ τὴν προτέραν ἧσσαν φοβουμένους
καὶ οὐ προθύμους ὄντας παρεκελεύσαντο. \\
\end{greek}


And for six or seven days they were lying at anchor opposite each other,
practicing and preparing for battle,
the Peloponnesians having the intention not to sail outside the Rhions into the broad waters,
afraid oftheir former misfortune,
and the Athenians having the intention not to sail into the narrows,
thinking that battle in a little space was in the enemy's favor.
Then Cnemus and the other generals of the Peloponnesians,
wanting to have the engagement quickly, before any aid came from Athens, called together the troops
and, seeing that the majority of them were afraid because of their former defeat
and that they were not eager, exhorted them. \\

\end{document}
