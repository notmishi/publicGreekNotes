\documentclass{article}
\usepackage{fontspec}
\usepackage{polyglossia}

\setmainlanguage{english}
\setotherlanguage[variant=polytonic]{greek}
\newfontfamily\greekfont[Script=Greek]{Gentium Plus}

\begin{document}

\section{Preface}

Greek text is sourced from \emph{Athenaze Book II, Second Edition} (2003) \\
English translation is adapted from \emph{Athenaze Book II: Teacher's Handbook, Revised Edition} (1991)

\section{Chapter 28 $\beta$}

\subsection*{\textgreek{Ο ΑΠΟΛΛΩΝ ΤΟΝ ΚΡΟΙΣΟΝ ΣΩΙΖΕΙ}}
\subsection*{Apollo saves Croesus}

\begin{greek}
ὁ μὲν Κῦρος ἐποίεε ταῦτα, ὁ δὲ Κροῖσος ἑστηκὼς ἐπὶ τῆς πυρῆς,
καίπερ ἐν Κακῷ ἐὼν τοσούτῳ, ἐμνήσθη τὸν τοῦ Σόλωνος λόγον,
ὅτι οὐδεὶς τῶν ζώντων εἴη ὄλβιος.
ὡς δὲ τοῦτο ἐμνήσθη ἀναστενάξᾶς ἐκ πολλῆς ἡσυχίης τρὶς ὠνόμασε, “Σόλων.”
καὶ Κῦρος ἀκούσᾱς ἐκέλευσε τοὺς ὲρμηνὲᾶς ἐρέσθαι τὸν Κροῖσον τίνα τοῦτον ἐπικαλέοιτο.
Κροῖσος δὲ πρῶτον μὲν σῑγὴν εἶχεν ἐρωτώμενος, τέλος δὲ ὡς ἠναγκάζετο,
εἶπε ὅτι ἦλθε παρ’ ἑαυτὸν ὁ Σόλων ἐὼν Ἀθηναῖος,
καὶ θεησάμενος πάντα τὸν ἑαυτοῦ ὄλβον περὶ οὐδενὸς ἐποιήσατο,
καὶ αὐτῷ πάντα ἀποβεβήκοι ᾗπερ ἐκεῖνος εἶπεν. \\
\end{greek}


That is what Cyrus did, but Croesus, standing on the pyre,
although he was in such great trouble, remembered the words of Solon,
that none of the living is happy.
When he remembered this, he groaned aloud,
and three times from the deep silence he called the name ``Solon.''
And Cyrus hearing this told his interpreters to ask Croesus who this was he was calling on.
And at first when Croesus was asked he kept silence,
but finally when he was forced, he said that Solon, an Athenian, had come to him
and after seeing all his wealth had considered it worthless,
and that everything had turned out for him as Solon had said. \\


\begin{greek}
ὁ μὲν Κροῖσος ταῦτα ἐξηγήσατο, τῆς δὲ πυρῆς ἤδη ἁμμένης ἐκαίετο τὰ ἔσχατα.
καὶ ὁ Κῦρος ἀκούσας τῶν ἑρμηνέων ἃ Κροῖσος εἶπε,
μεταγνούς τε καὶ ἐνθυμεόμενος ὅτι καὶ αὐτὸς ἄνθρωπος ἐὼν ἄλλον ἄνθρωπον,
γενόμενος ἑαυτοῦ εὐδαιμονίῃ οὐκ ἐλάσσονα, ζῶντα πυρῇ διδοίη,
καὶ ἐπιστάμενος ὅτι οὐδὲν εἴη τῶν ἐν ἀνθρώποις ἀσφαλές,
ἐκέλευσε σβεννύναι ὡς τάχιστα τὸ καιόμενον πῦρ
καὶ καταβιάζειν Κροῖσόν τε καὶ τοὺς μετὰ Κροίσου.
καὶ οἱ πειρώμενοι οὐκ ἐδύναντο ἔτι τοῦ πυρὸς ἐπικρατῆσαι. \\
\end{greek}


Croesus told this story, and the pyre had already been lit, and the furthest parts were burning.
And Cyrus, hearing from his interpreters what Croesus had said, changed his mind
and, pondering that he who was himself a man, was giving another man,
who had been no less than himself in his good fortune, alive to the fire,
and knowing that in human affairs nothing was safe, told his men to put out the burning fire as quickly as possible
and to bring down Croesus and those with Croesus.
And those who tried could not any longer get control over the fire. \\ \\ \\

\begin{greek}
ἐνταῦθα λέγεται ὑπὸ τῶν Λυδῶν τὸν Κροῖσον, μαθόντα τὴν Κύρου μετάγνωσιν,
βοῆσαι τὸν Ἀπόλλωνα, καλέοντα παραστῆναι καὶ σῶσαί μιν ἐκ τοῦ παρεόντος κακοῦ·
τὸν μὲν δακρύοντα ἐπικαλέεσθαι τὸν θεόν,
ἐκ δὲ αἰθρίης καὶ νηνεμίης συνδραμεῖν ἐξαίφνης νεφέλας,
καὶ χειμῶμά τε γενέσθαι καὶ πολὺ ὕδωρ, σβεσθῆναί τε τὴν πυρήν.
οὕτω δὴ μαθόντα τὸν Κῦρον ὡς εἴη ὁ Κροῖσος καὶ θεοφιλὴς καὶ ἀνὴρ ἀγαθός, ἐρέσθαι τάδε·
“Κροῖσε, τίς σε ἀνθρώπων ἔπεισε ἐπὶ γῆν τὴν ἐμὴν στρατευόμενον πολέμιον ἀντὶ φίλου ἐμοὶ καταστῆναι;”
ὁ δὲ εἶπε· “ὦ βασιλεῦ, ἐγὼ ταῦτα ἔπρηξα τῇ σῇ μὲν εὐδαιμονίῃ, τῇ δὲ ἑμαυτοῦ κακοδαιμονίῃ·
αἴτιος δὲ τούτων ἐγένετο ὁ Ἑλλήνων θεὸς ἐπάρας ἐμὲ στρατεύεσθαι.
οὐδεὶς γὰρ οὕτω ἀνόητός ἐστι ὅστις πόλεμον πρὸ εἰρήνης αἱρέεται·
ἐν μὲν γὰρ τῇ εἰρήνῃ οἱ παῖδες τοὺς πατέρας θάπτουσι,
ἐν δὲ τῷ πολέμῳ οἱ πατέρες τοὺς παῖδας. ἀλλὰ ταῦτα δαίμονί που φίλον ἦν οὕτω γενέσθαι.”
ὁ μὲν ταῦτα ἔλεγε, Κῦρος δὲ αὐτὸν λύσας καθεῖσέ τε ἐγγὺς ἑαυτοῦ καὶ μεγάλως ἐτίμα. \\
\end{greek}


Then it is said by the Lydians that Croesus, learning of Cyrus' change of mind, shouted for Apollo,
calling him to stand by him and save him from his present trouble.
Croesus called the god in tears, and, from a clear sky and windless calm,
clouds suddenly gathered, and a storm broke out and much rain, and the pyre was put out.
So Cyrus learned that Croesus was both dear to the gods and a good man,
and he asked him this, ``Croesus, what man persuaded you to march against my land and become my enemy instead of my friend?''
And he said, ``O king, Ι did this to your good luck and to my bad luck;
and the god of the Greeks is responsible for these things, who urged me to wage war.
For no one is so foolish that he chooses war in preference to peace;
for in peace sons bury their fathers, but in war fathers bury their sons.
But it was god's will  that this should happen, Ι suppose.''
So he spoke, and Cyrus freed him and made him sit down near him and honored him greatly.

\end{document}
