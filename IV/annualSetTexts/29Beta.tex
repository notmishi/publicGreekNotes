% vimTeX root = .
\documentclass{article}
\usepackage{fontspec}
\usepackage{polyglossia}

\setmainlanguage{english}
\setotherlanguage[variant=polytonic]{greek}
\newfontfamily\greekfont[Script=Greek]{Gentium Plus}

\begin{document}

\section{Preface}

Greek text is sourced from \emph{Athenaze Book II, Second Edition} (2003) \\
English translation is adapted from \emph{Athenaze Book II: Teacher's Handbook, Revised Edition} (1991)

\section{Chapter 29\textgreek{β}}

\subsection*{\textgreek{ΜΕΓΑ ΤΟ ΤΗΣ ΘΑΛΑΣΣΗΣ ΚΡΑΤΟΣ}}
\subsection*{Great is the Power of the Sea}

\begin{greek}
ὡς δὲ τό τε πνεῦμα κατῄει καὶ αἱ νῆες, ἐν ὀλίγῳ ἤδη οὖσαι,
ὑπὸ τοῦ τ᾿ ἀνέμου καὶ τῶν πλοίων ἅμα ἐταράσσοντο, καὶ ναῦς τ᾿ νηῒ προσέπιπτε,
οἱ δὲ ναῦται βοῇ τε χρώμενοι και λοιδορίᾳ οὐδὲν ἤκουον τῶν παραγγελλομένων, τότε δὴ σημαίνει ὁ Φορμίων·
καὶ οἱ Ἀθηναῖοι προσπεσόντες πρῶτον μὲν καταδύουσι τῶν στρατηγίδων νεῶν μίαν, ἔπειτα δὲ καὶ τὰς ἄλλας ᾗ χωρήσειαν διέφθειρον,
καὶ κατέστησαν αὐτοὺς ἐς φόβον, ὥστε φεύγουσιν ἐς Πάτρας καὶ Δύμην τῆς Ἀχαΐας.
οἱ δὲ Ἀθηναῖοι διώξαντες καὶ ναῦς δώδεκα λαβόντες τούς τε ἄνδρας ἐξ αὐτῶν τοὺς πλείστους ἀνελόμενοι,
ἐς Μολύκρειον ἀπέπλεον, καὶ τροπαῖον στήσαντες ἐπὶ τῷ Ῥίῳ ἀνεχώρησαν ἐς Ναύπακτον. \\
\end{greek}


When the breeze came down, and the ships, which were already in a confined space, were thrown into confusion at once by the wind and by the boats,
and ship fell against ship, and the sailors shouting and abusing each other\footnote{I think in-class we translated this as `shouting abuse at each other' instead}
heard none of the orders that were being passed along,
then Phormio gave the signal; and the Athenians falling on them first sink one of the flagships and then destroyed the others wherever they went,
and put them into a panic, so that they flee to Patrae and Dyme in Achaea.
And the Athenians chased them and took twelve ships and picked up most of the men from them;
then they sailed away toward Moly- crion,
and after setting up a trophy at Rhion they withdrew to Naupactus. \\


\begin{greek}
παρέπλευσαν δὲ καὶ οἱ Πελοποννήσιοι εὐθὺς ταῖς περιλοίποις τῶν νεῶν ἐκ τῆς Δύμης καὶ Πατρῶν ἐς Κυλλήνην.
καὶ ἀπὸ Λευκάδος Κνῆμός τε καὶ αἱ ἐκείνων νῆες ἀφικνοῦνται ἐς τὴν Κυλλήνην.
πέμπουσι δὲ καὶ οἱ Λακεδαιμόνιοι τῷ Κνήμῳ ξυμβούλους ἐπὶ τᾶς ναῦς,
κελεύοντες ἄλλην ναυμαχίαν βελτίονα παρασκευάζεσθαι καὶ μὴ ὑπ᾿ ὀλίγων νεῶν εἴργεσθαι τῆς θαλάσσης.
οὐ γὰρ ᾤοντο σφῶν τὸ ναυτικὸν λείπεσθαι, ἀλλὰ γεγενῆσθαί τινα μαλακίαν·
ὀργῇ οὖν ἀπέστελλον τοὺς ξυμβούλους. οἱ δὲ μετὰ τοῦ Κνήμου ἀφικόμενοι ἄλλας τε ναῦς μετεπέμψαντο,
τοὺς ξυμμάχους παρακαλοῦντες βοηθεῖν, καὶ τὰς προϋπαρχούσας ναῦς ἐξηρτύοντο ὡς ἐπὶ μάχην. \\
\end{greek}


The Peloponnesians sailed along with the rest of their ships straight from Dyme and Patrae to Cyllene.
And Cnemus and the ships of the Leucadians arrive at Cyllene from Leucas.
And the Spartans also send advisers for Cnemus to the fleet,
telling him to prepare another and more successful sea battle and not to be shut out from the sea by a few ships.
For they did not think that their fleet was deficient but that some cowardice had occurred;
and so they sent off the advisers in anger.
And those who had come with Cnemus sent for other ships, summoning their allies to help,
and they fitted out the ships already there for battle. \\


\begin{greek}
πέμπει δὲ καὶ ὁ Φορμίων ἐς τὰς Ἀθήνας ἀγγέλους τήν τε παρασκευὴν αὐτῶν ἀγγελοῦντας
καὶ περὶ τῆς ναυμαχίας ἣν ἐνίκησαν φράσοντας,
καὶ κελεύων αὐτοὺς ἑαυτῷ ναῦς ὡς πλείστας ταχέως ἀποστεῖλαι,
ὡς καθ᾿ ἡμέραν ἐλπίδος οὔσης ναυμαχήσειν.
οἱ δὲ Ἀθηναῖοι πέμπουσιν εἴκοσι ναῦς αὐτῷ,
τῷ δὲ κομίζοντι αὐτὰς προσεπέστειλαν ἐς Κρήτην πρῶτον ἀφικέσθαι,
ἵνα ξυμμάχοις τισὶν ἐκεῖ βοηθοίη. \\
\end{greek}


And Phormio also sends messengers to Athens, to announce the enemy's preparations
and to tell of the battle that they had won,
and telling them to quickly send off to him as many ships as possible,
as he expected every day to fight a naval battle.
And the Athenians send him twenty ships and instructed the man bringing them in addition
to go to Crete first to help some allies there.

\end{document}
