% vimTeX root = .
\documentclass{article}
\usepackage{fontspec}
\usepackage{polyglossia}

\setmainlanguage{english}
\setotherlanguage[variant=polytonic]{greek}
\newfontfamily\greekfont[Script=Greek]{Gentium Plus}

\begin{document}

\section{Preface}

Greek text is sourced from \emph{Athenaze Book II, Second Edition} (2003) \\
English translation is adapted from \emph{Athenaze Book II: Teacher's Handbook, Revised Edition} (1991)

\section{Chapter 29\textgreek{δ}}

\subsection*{\textgreek{ΜΕΓΑ ΤΟ ΤΗΣ ΘΑΛΑΣΣΗΣ ΚΡΑΤΟΣ}}
\subsection*{Great is the Power of the Sea}

\begin{greek}
οἱ δὲ Πελοποννήσιοι, ἐπειδὴ αὐτοῖς οἱ Ἀθηναῖοι οὐκ ἐπέπλεον ἐς τὸν κόλπον,
βουλόμενοι ἄκοντας ἔσω προαγαγεῖν αὐτούς, ἀναγαγόμενοι ἅμα ἕῳ ἔπλεον ἐπὶ τοῦ κόλπου,
ἐπὶ τεσσάρων ταξάμενοι τὰς ναῦς δεξιῷ κέρᾳ ἡγουμένῳ, ὥσπερ καὶ ὥρμουν·
ἐπὶ δὲ τούτῳ τῷ κέρᾳ εἴκοσι ἔταξαν τὰς ναῦς τὰς ἄριστα πλεούσας,
ἵνα, εἰ ὁ Φορμίων, νομίσας ἐπὶ τὴν Ναύπακτον αὐτοὺς πλεῖν, ἐπιβοηθῶν ἐκεῖσε παραπλέοι,
μὴ διαφύγοιεν τὸν ἐπίπλουν σφῶν οἱ Ἀθηναῖοι, ἀλλὰ αὗται αἱ νῆες περικλῄσειαν. \\
\end{greek}


The Peloponnesians, when the Athenians did not sail into the gulf against them,
wanting to lead them into the gulf against their will, put out to sea at dawn
and sailed in the direction of the gulf, arranging their ships four deep, with the right wing leading,
just as they had been at anchor; and on this wing they posted their twenty fastest sailing ships,
so that, if Phormio thought that they were sailing against Naupactus and sailed there to help,
the Athenians would not escape their attack, but these ships would shut them in. \\


\begin{greek}
ὁ δὲ Φορμίων, ὅπερ ἐκεῖνοι προσεδέχοντο, φοβηθεὶς περὶ τῷ χωρίῳ ἐρήμῳ ὄντι,
ὡς ἑώρα ἀναγομένους αὐτούς, ἄκων καὶ κατὰ σπουδὴν ἐμβιβάσας, ἔπλει παρὰ τὴν γῆν·
καὶ ὁ πεζὸς στρατὸς ἅμα τῶν Μεσσηνίων παρεβοήθει.
ἰδόντες δὲ οἱ Πελοποννήσιοι αὐτοὺς κατὰ μίαν παραπλέοντας
καὶ ἤδη ὄντας ἐντὸς τοῦ κόλπου τε καὶ πρὸς τῇ γῇ, ὅπερ ἐβούλοντο μάλιστα,
ἀπὸ σημείου ἑνὸς εὐθὺς ἐπιστρέψαντες τὰς ναῦς μετωπηδὸν ἔπλεον ὡς τάχιστα ἐπὶ τοὺς Ἀθηναίους,
καὶ ἤλπιζον πάσας τὰς ναῦς ἀπολήψεσθαι. \\
\end{greek}


And Phormio, just as they were expecting, frightened for the place which was deserted,
when he saw them putting out to sea, reluctantly and hastily embarked and sailed along the land,
and at the same time the infantry of the Messenians came to their aid.
And the Peloponnesians, seeing them sailing along in single file
and already inside the gulf and close to land, which they had wanted most,
at one signal immediately turned their ships
and sailed in close line with all speed against the Athenians,
and hoped to cut off all the ships. \\


\begin{greek}
τῶν δὲ Ἀθηναίων νεῶν ἕνδεκα μὲν αἵπερ ἡγοῦντο ὐπεκφεύγουσι τὸ κέρας τῶν Πελοποννησίων·
τὰς δὲ ἄλλας καταλαβόντες οἱ Πελοποννήσιοι ἐξέωσάν τε πρὸς τὴν γῆν ὑπεκφευγούσας καὶ διέφθειραν·
ἄνδρας τε τῶν Ἀθηναίων ἐπέκτειναν ὅσοι μὴ ἐξένευσαν αὐτῶν.
καὶ τῶν νεῶν τινας οἱ Μεσσήνιοι, παραβοηθήσαντες καὶ ἐπεσβαίνοντες ξὺν τοῖς ὅπλοις ἐς τὴν θάλασσαν καὶ ἐπιβάντες,
ἀπὸ τῶν καταστρωμάτων μαχόμενοι ἀφείλοντο ἑλκομένας ἤδη. \\
\end{greek}



But eleven of the Athenian ships, which were leading, tried to escape the wing of the Peloponnesians;
but the others the Peloponnesians caught and pushed out toward the land as they tried to escape and disabled  them.
And they killed all the Athenians who did not swim to shore.
And they took in tow some of the ships and pulled them empty (and one they had already taken with its crew),
but the Messenians, who came to help and went into the sea with their weapons,
boarded and, fighting from the decks, saved some when they were already being towed away.

\end{document}
